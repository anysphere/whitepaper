\section{Introduction}
\label{sec:introduction}

% When the internet was first established, everything sent over it was public. 
% If A sent a message to B, anyone on their path\stzh{route?} through the internet could see that such a message was sent, as well as read the actual message. As of today, many messaging services are end-to-end encrypted, meaning that no one can read the contents of messages. However, for sufficiently powerful adversaries — hackers, ISPs, government agencies — it is still possible to find out who is talking to whom, as well as when and how often messages are sent. Our goal is to hide this metadata: we want to create a system where A and B can send messages to each other over an untrusted network, without anyone knowing that they talk to each other. Such a private communication system would be critically important to protect and expand freedom in the world \cite{arvid}.

% From the start, Anysphere has assumed a single important security principle: \textit{no needless trust}. This simple principle guide Anysphere's development and is the key motivator behind our architecture, our threat model, and our choice of private information retrieval as our core protocol. 

% Quite simply, we believe that communication must place no needless trust in systems across the internet since many conversations are so meaningful\todo{impactful?} that unwarranted trust can lead to harm. In particular, we currently only trust the local device and your friends. That is it.

% Our principles are fundamentally different from Signal, Email, and other platforms and protocols that assume trust in the servers, in the plumbing of the internet, and in networks being so large that no entity can monitor them. We also solve many of the practical problems that plague research prototypes. We are working hard to make Anysphere more secure, and we are working hard to make it extremely security realistic for everyone.

% Easter egg: "The best way to keep a secret is to never have it." - Bruce Schneier

% PLAN:

% paragraph 1: motivation

% what to convey?

% why should i care about metadata privacy?
    % can reveal a lot of information (quantify this? make it practical)
% what is metadata privacy?
    % ok this needs to go in first sentence
% start from the point of view of end-to-end encryption
    % this is important

% vibe:
% no alarmism please
% bitcoin whitepaper is... SO GOOD

% paragraph 2: what is needed is... no needless trust.

% what are you doing? (no needless trust)
% is what you are doing a paradigm shift? (yes)
% is this unique? (yes)


Electronic communication runs the world. Yet, it is not as secure or private as the in-person communication it replaced. Advances such as end-to-end encryption are great for protecting \textit{what} is being said, but information about who is talking to whom, how often and when — \textit{the metadata} — is still being leaked at scale. Hackers, social media companies, and malicious nation states have open access to the metadata, even for individuals, organizations and governments that use the most secure communication platforms on the market.

%Electronic communication runs the world. Yet, it is not as secure or private as the in-person communication it replaced. Advances such as end-to-end encryption are great for protecting \textit{what} is being said, but the \textit{metadata}, who is talking to whom, how often and when, is still being leaked at scale. Hackers, social media companies, and malicious nation states can have open access to the metadata even with the most secure communication platforms on the market.

%What is needed is a new approach: the content and \textit{metadata} of every conversation must be protected. The best way to protect conversation data is not to give anyone access to it in the first place. Therefore, Anysphere operates on the principle of \textit{no needless trust}. Even if our server's are compromised, everyone's communication history and pattern would still be secure.

What is needed is a new approach: every part of every conversation must be protected, including all metadata. The best way to protect something is to not give anyone access to it in the first place. Therefore, Anysphere operates on the principle of \textit{no needless trust}. Even if all of our servers are compromised, everyone's communication history and pattern would still be secure.

This whitepaper explains how Anysphere works. Our code is open source and available at$~${\tt \href{https://github.com/anysphere/client}{github.com/anysphere/client}}.