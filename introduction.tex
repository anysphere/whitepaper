\section{Introduction}
\label{sec:introduction}

When the internet was first established, everything sent over it was public. 
If A sent a message to B, anyone on their path\stzh{route?} through the internet could see that such a message was sent, as well as read the actual message. As of today, many messaging services are end-to-end encrypted, meaning that no one can read the contents of messages. However, for sufficiently powerful adversaries — hackers, ISPs, government agencies — it is still possible to find out who is talking to whom, as well as when and how often messages are sent. Our goal is to hide this metadata: we want to create a system where A and B can send messages to each other over an untrusted network, without anyone knowing that they talk to each other. Such a private communication system would be critically important to protect and expand freedom in the world \cite{arvid}.

From the start, Anysphere has assumed a single important security principle: \textit{no needless trust}. This simple principle guide Anysphere's development and is the key motivator behind our architecture, our threat model, and our choice of private information retrieval as our core protocol. 

Quite simply, we believe that communication must place no needless trust in systems across the internet since many conversations are so meaningful\todo{impactful?} that unwarranted trust can lead to harm. In particular, we currently only trust the local device and your friends. That is it.

Our principles are fundamentally different from Signal, Email, and other platforms and protocols that assume trust in the servers, in the plumbing of the internet, and in networks being so large that no entity can monitor them. We also solve many of the practical problems that plague research prototypes. We are working hard to make Anysphere more secure, and we are working hard to make it extremely security realistic for everyone.
