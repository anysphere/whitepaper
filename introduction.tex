\section{Introduction}

When the internet was first established, everything sent over it was public. If A sent a message to B, anyone on their path through the internet could see that such a message was sent, and read the actual message. As of today, many messaging services are end-to-end encrypted, meaning that no one can read the contents of messages. However, for sufficiently powerful adversaries, it is still possible to find out who is talking to whom. The goal of private communication is to hide this metadata: we want to create a system where A and B can send messages to each other over an untrusted network, without anyone knowing that they talk to each other. Such a private communication system would be critically important to protect and expand freedom in the world \cite{arvid}.

Anysphere is a provably secure metadata-hiding communication platform, built for the real world. In this document, we outline how we guarantee theoretical security, as well as how we make sure that the theoretical security translates to good practical security.
