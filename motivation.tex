% \section{Introduction}

\section{Motivation}

% When the internet was first established, everything sent over it was public. If A sent a message to B, anyone on their path through the internet could see that such a message was sent, as well as read the actual message. As of today, many messaging services are end-to-end encrypted, meaning that no one can read the contents of messages. However, for sufficiently powerful adversaries — hackers, ISPs, government agencies — it is still possible to find out who is talking to whom, as well as when and how often messages are sent. Our goal is to hide this metadata: we want to create a system where A and B can send messages to each other over an untrusted network, without anyone knowing that they talk to each other. Such a private communication system would be critically important to protect and expand freedom in the world \cite{arvid}.

% Metadata-private communication has been studied for decades. In 1981, David Chaum introduced so-called mix-nets, which are

% In the 1980s, David Chaum introduced DC-nets \cite{chaum1988dining}, showing that it is possible for three people to exchange information without having to reveal from whom the information is coming from. This was long believed to be too unpractical for actual use. hi hi

% While there has been a lot of research focusing on the theoretical problem of message transmission, there has been less attention on everything else that needs to exist for a communication system to be useful in practice: initiating connections, handling arbitrary failures, and distributing code securely, to name a few. We draw on the knowledge of the past for message transmission and invent novel protocols for the rest, some of which may be of interest to the research community.