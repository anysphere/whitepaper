\subsection{Threat Model}
\begin{figure}
\begin{minipage}[t]{0.3333\textwidth}
\textbf{Non-E2E}
\begin{framed}
Bob,

Remember 2.1.1978?

Alice. 

06.01.2022, 10:30 PM.
\end{framed}
\end{minipage}%
\begin{minipage}[t]{0.3333\textwidth}
\textbf{E2E}
\begin{framed}
Bob,

xxxxxxxxxxxxxxxxxx

Alice. 

06.01.2022, 10:30 PM.
\end{framed}
\end{minipage}%
\begin{minipage}[t]{0.3333\textwidth}
\textbf{Metadata Hiding}
\begin{framed}
xxxxxxxxxxxxxxxxxx

xxxxxxxxxxxxxxxxxx

xxxxxxxxxxxxxxxxxx

xxxxxxxxxxxxxxxxxx
\end{framed}
\end{minipage}
\caption{What a message Alice sent Bob looks like for a powerful adversary controlling all servers, under non-E2E, E2E, and Metadata-Hiding protocols.}
\end{figure}
%\begin{table*}[t]
%\centering
%another version below
%\begin{tabular}{||c c c c c c||} 
% \hline
%  Attacker compromises $\cdots$ & Anysphere & Signal & Skiff & Wickr & Onion  \\
% \hline
% Attacker listens on the internet & \checkmark & \checkmark & \checkmark & \checkmark \\ 
% \hline
% Attacker compromises & \checkmark & & & \checkmark & \checkmark \\
% \hline
% Skiff & \checkmark &  &  & & \checkmark\\
% \hline
% Wickr & \checkmark & & & \checkmark & \checkmark\\
% \hline
% Onion & \checkmark & *\footnote{\label{onion}Only guaranteed for non-global adversary} & &\checkmark&\checkmark\\
% \hline
%\end{tabular}
%\caption{Comparing when}
%\end{table*}


\xxx{add an illustration of a walled garden, with the walls containing only your computer and your friends' computers}

%Touch on: server, friends, client-side computer, etc.

\xxx{Figure out a better format here. See e.g. Skiff's whitepaper.}

1. \textbf{Compromised Server}: We do not have any trust in the server. Similar to most PIR schemes(for example \cite{ahmad2021addra}, \textsection 2.2), our threat model assumes a global adversary who can compromise the entire communication infrastructure except for the user's and their friends' client end. In particular, we assume the adversary has control over all the servers, and can observe and manipulate all network traffic.

2. \textbf{Compromised Strangers}: Similarly, we assume that the adversary has control over all non-friend clients, and can observe and manipulate all network traffic from and to these clients.

3. \textbf{Trusted Friends}: In \cite{angel2018s}, Angle, Lazar and Tzialla describes the compromised friend(CF) attack on a general meta-data private messaging system, which shows that perfectly hiding metadata while not trusting the user's friends is computationally prohibitive. In our threat model, a user trusts that their devices, as well as the devices of all their friends, are uncompromised and running our client code. \xxx[stzh]{In the future, we would look into measures for handling friend compromises?} 

3. \textbf{Secure Cryptographical Primitives}: Our threat model assumes the security of the standard cryptography primitives we use, including Microsoft SEAL's BFV cryptosystem and libsodium's AEAD cryptosystem. 



\subsection{Non-goals}
1. \textbf{Not a cryptocurrency}: Anysphere's underlying PIR algorithm is unrelated to blockchains and cryptocurrencies. \xxx[stzh]{this is, unfortunately, what most people think of when they here ``private communication".}

2. \textbf{Not a plugin}: Anysphere relies on a dedicated PIR server to work. Therefore, Anysphere is not compatible with existing messaging systems like Messenger or Signal, and does not work like a google chrome extension to these apps. \xxx[stzh]{My mom asked me this...}

3. \textbf{Not steganographic}: Anysphere does not make an attempt to hide who's using our service. \xxx[stzh]{From Tor. Is it ok?} 
\xxx[sualeh] {I like both of these.}

