\subsection{Threat Model}
\label{subsec:threatmodel}

This section defines the capabilities that we assume an attacker has. Our belief in no needless trust means that our threat model is as extensive as possible.

%\begin{table*}[t]
%\centering
%another version below
%\begin{tabular}{||c c c c c c||} 
% \hline
%  Attacker compromises $\cdots$ & Anysphere & Signal & Skiff & Wickr & Onion  \\
% \hline
% Attacker listens on the internet & \checkmark & \checkmark & \checkmark & \checkmark \\ 
% \hline
% Attacker compromises & \checkmark & & & \checkmark & \checkmark \\
% \hline
% Skiff & \checkmark &  &  & & \checkmark\\
% \hline
% Wickr & \checkmark & & & \checkmark & \checkmark\\
% \hline
% Onion & \checkmark & *\footnote{\label{onion}Only guaranteed for non-global adversary} & &\checkmark&\checkmark\\
% \hline
%\end{tabular}
%\caption{Comparing when}
%\end{table*}


% \xxx{add an illustration of a walled garden, with the walls containing only your computer and your friends' computers} 

%Touch on: server, friends, client-side computer, etc.

\textbf{1. The attacker may compromise all servers.} To achieve privacy, we do not put any trust in the server. Our threat model assumes a global adversary who has full control over all servers, and can observe and manipulate all network traffic. This is similar to other anonymous communication schemes based on private information retrieval (for example \cite{ahmad2021addra}, \textsection 2.2), but stronger than most other anonymous communication schemes (e.g. Tor \cite{dingledine2004tor} and Nym \cite{piotrowska2017loopix}, which require partial trust in the servers).

\textbf{2. The attacker may observe and control the entire internet.} See above.

\textbf{3. The attacker may compromise strangers.}: We assume that the attacker has control over all clients that are not a contact of a given user, and can send maliciously crafted messages to and from these clients.

\textbf{4. The attacker cannot compromise contacts.}: We assume that a user's contacts are trusted, and that the attacker does not have access to their computers. In \cite{angel2018s}, Angel, Lazar and Tzialla describes an attack on a general metadata-private communication system, which shows that perfectly hiding metadata while not trusting the user's contacts is computationally prohibitive. The amount of metadata leaked to contacts is small, and we plan to look into measures for handling compromised contacts in the future (see \cref{sec:future}).

\textbf{5. The attacker does not have access to the user's computer.} We assume that a user's local computer is trusted and is running a correct implementation of our system. In \cref{sec:practical-security} we explain how we can relax this assumption slightly.

\textbf{6. The attacker cannot break standard cryptography.}: Our threat model assumes the security of the standard cryptography primitives we use. This includes Libsodium's AEAD implementation (XSalsa20), and Microsoft SEAL's homomorphic encryption implementation (BFV).



\subsection{Non-goals}
\textbf{1. Not a cryptocurrency.} Anysphere uses advanced cryptography, but not blockchains or cryptocurrencies.

\textbf{2. Not a plugin to an existing ecosystem.} Anysphere is not compatible with existing messaging systems like Signal or Email. This is intentional: interfacing with legacy systems would mean accepting their (much lower) standards of security and privacy.

\textbf{3. Not steganographic.} Anysphere does not make an attempt to hide who's using our service.

