\section{What's Next?}
\label{sec:future}

Private communication is a hard problem. We are actively doing research to increase convenience while preserving metadata privacy, and we urge the research community to join us.

% Anysphere has just begun operations, and we are devoted to truly free communication. 
% We aim to supervise and help build technologies that allow internet communication without unfair security assumptions.
% There are several important milestones that we have set for ourselves to reach to make this genuinely possible; some that are essential in the short-term and others that we want to achieve over a longer time horizon with time and effort.

\subsection{Implementation milestones}
Anysphere is under active development. This is a list of some of the things we are working on.

\textbf{Small files and images.} We hope to support the transmission of small files and images through our current protocol.

\textbf{Deleting messages.} We plan to implement support for deleting messages, making sure that the messages are deleted forever with a ratcheting mechanism that guarantees forward secrecy.

\textbf{Public-key infrastructure.} We hope to set up a PKI to help the trust establishment process. That said, setting up a PKI is a notoriously hard problem requiring care.

\textbf{Calls.} We would like to facilitate calls using Anysphere without leaking metadata. However, calls using our current PIR setup require infeasible sub-second latency.

\textbf{Hardened daemon against local malware.} We aim to reduce the amount of trust we place in the client (see \cref{sec:practical-security}), by hardening the daemon.

\subsection{Research problems} 
A lot of the problems we want to solve are both hard and unsolved. We're actively conducting research on these problems.

\textbf{Denial-of-service resistance.} While we fundamentally cannot prevent an internet provider from blocking access, we can reduce the trust assumptions when it comes to denial of service. For example, it would be great if we could distribute the servers, and provide a guarantee in the form of ``the attacker needs to compromise at least $k$ of $n$ servers.''

\textbf{One-to-many and many-to-many conversations.} We want to allow groups of people to broadcast messages to each other, without depending on the presence of any specific user. The question is how to do it scalably, while also embodying \textit{no needless trust}.

\textbf{More efficient trust establishment.} We are actively thinking about improving our current methods of trust establishment to be faster, more reliable and less prone to social engineering attacks.

\textbf{The ACK problem.} Currently, if Alice wants to send a 2 KB message to Bob, she needs to wait for Bob to get online and ACK the first chunk before she can send the second chunk. If Alice's and Bob's time zones do not overlap, this can mean that an $n$ KB message takes $n$ days to get delivered.

\textbf{The multiple contacts problem.} In our current protocol, the transmission time for a message scales linearly with the number of contacts a user has. We also want to increase security by handling compromised contacts better.

\textbf{Large files and images.} Transmission of larger files and images is infeasible in our current framework because the number of chunks needed to deliver them is in the thousands.
