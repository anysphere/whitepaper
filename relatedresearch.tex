\section{Related Research}

Metadata-private communication has been studied for decades. In 1981, David Chaum introduced \textbf{mix-nets}, which bounce messages between a small number of servers. Combined with onion encryption, mix-nets makes it impossible to determine the destination of a given source packet. However, mix-nets require the assumption that at least one server is honest. Using mix-nets, Tor was created in 2002, and became one of the most successful privacy-protecting real-world projects \cite{dingledine2004tor}. Unfortunately, in addition to the server trust issue, mix-nets leak timing data which makes it easy for someone with ISP-level control of the network to observe who is talking to whom. In today's world, it is getting easier and easier to amass enough data to perform such correlation attacks, making mix-net based approaches unsuitable for real security \cite{karunanayake2021anonymisation}.

\xxx[stzh]{I believe passive voice should be avoided.}

The 2010s saw a flurry of research papers attempting a few different methods to achieve scalable metadata-privacy: Dissent and Riposte \cite{corrigan2010dissent,corrigan2015riposte} introduced \textbf{DC-nets}, Vuvuzela, Atom, Talek and many others \cite{van2015vuvuzela,cheng2020talek,kwon2017atom} introduced mix-nets with stronger security guarantees, Clarion, mcMix and Blinder \cite{alexopoulos2017mcmix,eskandarian2021clarion,abraham2020blinder} introduced multi-party computation techniques, and NIAR \cite{shi2021non,bunz2021non} introduced function-hiding functional encryption. All but NIAR's approach are less secure than our PIR-based approach; NIAR is impractical at scale due to computation time. The PIR line of work, started by Angel with Pung \cite{angel2016unobservable,angel2018pir} and continued by Addra \cite{ahmad2021addra}, is the only one that promises both perfect security and reasonable scalability.

While there has been a lot of research focusing on the theoretical problem of message transmission, there has been less attention on everything else that needs to exist for a communication system to be useful in practice: initiating connections, handling arbitrary failures, and distributing code securely, to name a few. We draw on the knowledge of the past for message transmission and invent novel protocols for the rest, some of which may be of interest to the research community.