\section{Related Research}

The research community has studied metadata-private communication for decades. In 1981, David Chaum introduced \textit{mix-nets}, which bounce messages between a small number of servers \cite{chaum1981untraceable}. Combined with onion encryption, mix-nets make it impossible to determine the destination of a given source packet, \textit{assuming that at least one server is honest}. Using mix-nets, Tor was created in 2002 and became one of the most successful privacy-protecting real-world projects \cite{dingledine2004tor}. Unfortunately, in addition to the server trust issue, mix-nets leak timing data, making it easy for someone with ISP-level network control to observe who is talking to whom. In today's world, it is getting easier and easier to amass enough data to perform such correlation attacks, making mix-net-based approaches unsuitable \cite{karunanayake2021anonymisation}.

Systems with stronger metadata-privacy guarantees saw a flurry of interest in the last decade: Dissent and Riposte  used so-called DC-nets \cite{corrigan2010dissent,corrigan2015riposte}, Vuvuzela, Atom, Talek and many others  introduced mix-nets with stronger security guarantees \cite{van2015vuvuzela,cheng2020talek,kwon2017atom}, Clarion, mcMix and Blinder introduced multi-party computation techniques \cite{alexopoulos2017mcmix,eskandarian2021clarion,abraham2020blinder}, and NIAR  introduced function-hiding functional encryption \cite{shi2021non,bunz2021non}.
All but NIAR's approach are less secure than our PIR-based approach, and NIAR is impractical at scale due to computation time.
The PIR line of work, started by Angel with Pung \cite{angel2016unobservable,angel2018pir} and continued by Addra \cite{ahmad2021addra}, promises both perfect security and reasonable scalability.

A communication system needs more than just message transmission. There has been much less attention in the literature to other components: establishing trust, managing many-to-many conversations, and defending against denial of service, to name a few. For trust establishment, Liu and Tromer recently initiated work on oblivious message retrieval \cite{liutromer2021}, which is unfortunately not scalable enough for us, but a good start to a field we want to develop further.