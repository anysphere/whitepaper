\section{Related Research}

Cryptography researchers have been studying metadata-private communication for decades. In 1981, David Chaum introduced \textit{mix-nets}, which bounce messages between a small number of servers. Combined with onion encryption, mix-nets make it impossible to determine the destination of a given source packet, \textit{assuming that at least one server is honest}. Using mix-nets, Tor was created in 2002 and became one of the most successful privacy-protecting real-world projects \cite{dingledine2004tor}. Unfortunately, in addition to the server trust issue, mix-nets leak timing data, making it easy for someone with ISP-level network control to observe who is talking to whom. In today's world, it is getting easier and easier to amass enough data to perform such correlation attacks, making mix-net-based approaches unsuitable for absolute security \cite{karunanayake2021anonymisation}.

Systems with stronger metadata-privacy guarantees saw a flurry of interest in the last decade: Dissent and Riposte \cite{corrigan2010dissent,corrigan2015riposte} used so-called DC-nets, Vuvuzela, Atom, Talek and many others \cite{van2015vuvuzela,cheng2020talek,kwon2017atom} introduced mix-nets with stronger security guarantees, Clarion, mcMix and Blinder \cite{alexopoulos2017mcmix,eskandarian2021clarion,abraham2020blinder} introduced multi-party computation techniques, NIAR \cite{shi2021non,bunz2021non} introduced function-hiding functional encryption.
All but NIAR's approach are less secure than our PIR-based approach, and NIAR is impractical at scale due to computation time.
The PIR line of work, started by Angel with Pung \cite{angel2016unobservable,angel2018pir} and continued by Addra \cite{ahmad2021addra}, promises both perfect security and reasonable scalability.

A modern communication system depends on more than just message transmission. There has been much less attention in the literature to other components: establishing trust, handling arbitrary failures, and distributing code securely, to name a few. For trust establishment, Liu and Tromer recently initiated work on oblivious message retrieval (OMR) \cite{liutromer2021}, which is unfortunately not scalable enough for us, but a good start to a field we want to develop further.