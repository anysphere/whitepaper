\section{Related Research}

Metadata-private communication has been studied for decades. In 1981, David Chaum introduced so-called mix-nets, which bounce messages between a small number of servers. Using onion encryption, if at least one of the servers is honest, it is impossible to determine the destination of a given source packet. Tor, created in 2002, is one of the most successful privacy-protecting real-world projects, and uses mix-nets \cite{dingledine2004tor}. Unfortunately, even aside from the server trust issue, mix-nets leak timing data which makes it easy for someone with ISP-level control of the network to observe who is talking to whom. In today's world, it is getting easier and easier to amass enough data to perform such correlation attacks, making mix-net based approaches unsuitable for real security \cite{karunanayake2021anonymisation}.

The 2010s included a flurry of research papers trying out a few different methods of achieving scalable metadata-privacy: so-called DC-nets were tried by Dissent and Riposte \cite{corrigan2010dissent,corrigan2015riposte}, mix-nets with stronger security guarantees were tried by Vuvuzela, Atom, Talek and many others \cite{van2015vuvuzela,cheng2020talek,kwon2017atom}, multi-party computation techniques were tried by Clarion, mcMix and Blinder \cite{alexopoulos2017mcmix,eskandarian2021clarion,abraham2020blinder}, and function-hiding functional encryption was tried by NIAR \cite{shi2021non,bunz2021non}. These approaches are either less secure than the PIR-based approach we are using (all but NIAR), or are impractical at scale due to computation time (NIAR). The PIR line of work, started by Angel with Pung \cite{angel2016unobservable,angel2018pir} and continued by Addra \cite{ahmad2021addra}, is the only one that promises both perfect security and reasonable scalability.

While there has been a lot of research focusing on the theoretical problem of message transmission, there has been less attention on everything else that needs to exist for a communication system to be useful in practice: initiating connections, handling arbitrary failures, and distributing code securely, to name a few. We draw on the knowledge of the past for message transmission and invent novel protocols for the rest, some of which may be of interest to the research community.