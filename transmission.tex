\section{Core Protocol}

Alice wants to message Bob over an untrusted network, without leaking any data or metadata to anyone. To hide the message content they just use end-to-end encryption. To hide metadata they employ two simple key ideas: sending data at a constant rate, and retrieving homomorphically compressed data at a constant rate.

When signing up for Anysphere, each user gets their own \textit{outbox} on the server. This outbox is a dedicated storage space that the user sends messages to. Once every minute, Alice will send exactly 1 KB of data to her outbox on the server. If she has a message to send, she sends the encryption of that message, and if she has no message to send, she sends a random sequence of bytes. With this simple first idea, no one, including the server and any network observers, will know when Alice actually sends a message.

The message needs to be routed to Bob. Now, a traditional messaging system would have Bob download data from Alice's outbox, and then try to decrypt it to see if it was meant for him. But this leaks metadata: the server would know that Alice wrote to outbox $x$ and that Bob read from outbox $x$, which links the two of them together!

The simple solution is for Bob to, once every minute, download \textit{all} of the outboxes on the server. Locally, on his own computer, he can then check Alice's outbox. This way, the server, and no one else except Alice, has any way of linking Bob to Alice. All metadata is protected.

This is how Anysphere works. Obviously, Bob cannot download all outboxes every minute — that would be way too much data! — so instead he uses \textit{private information retrieval}, a well-studied cryptographic primitive, as a way of compressing his download size. The following subsections will describe the system in detail.

\subsection{Registration}

\subsection{Private information retrieval}

\subsection{Acknowledgements}

TCP/IP etc.