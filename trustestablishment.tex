\section{Trust Establishment}
\label{sec:trustestablishment}

Our core protocol requires users to share a secret key. Creating this shared secret is generally referred to as the problem of trust establishment \cite{unger2015sok}, and involves making sure that the person you talk to is who they say they are. We have an additional requirement that has only recently been explored by the research community \cite{liutromer2021}: the trust establishment procedure must not leak metadata.

%In practice, Alice needs to find Bob's public key and use it to send an ``Invitation" to Bob. Bob must be able to retrieve this invitation and complete a key exchange with Alice. Most importantly, the server should not learn any metadata of this exchange. 

We provide two different methods of trust establishment: one that completes instantly, but requires the two users to meet in person, and one that allows asynchronous invitations, but can take up to 24 hours. In both methods, we assume that Alice and Bob each have a Curve25519 key exchange key pair $\kx = (\kx^P, \kx^S)$, and aim to inform each other of their $\kx^P$ to derive a shared secret $k$. Alice and Bob additionally need to know each other's outbox indices.

% Note that we use the key exchange public key $\kx^P$ as the sole root of trust of a user's identity. This is intentional: using more human-readable identifiers like username or email opens the possibility of social engineering attacks, which we would like to avoid.

%\xxx[Sualeh]{Handle the part about why we do not include the username in the publicID.}

% \cite{liutromer2021} discusses this as the Oblivious Message Detection(OMD) problem. Their proposed scheme aims to minimize user download size, and costs each user $\$ 1$ per million messages scanned, which is prohibitive for our messaging application. We provide two alternate methods of trust establishment with better computational cost and security \footnote{We should not appear to be strictly better than \cite{liutromer2021}; any ideas?}.




%%%%%%%%%%%%%%%%%%%%%%%%%%%%%%%%%%%%%%%%%%%%%%%%%%%%%%%%%%%%%%%
%%%%% Sync Section %%%%%
%%%%%%%%%%%%%%%%%%%%%%%%%%%%%%%%%%%%%%%%%%%%%%%%%%%%%%%%%%%%%%%

% \captionsetup[figure]{name=Protocol}



\subsection{In-person Trust Establishment}

Alice and Bob establish trust in an in-person meeting, and do a simple key exchange where they each share their key-exchange public key $\kx^P$ and outbox index $i$ out loud. \Cref{fig:trust-establishment-inperson} explains the protocol in detail.

\begin{figure}[h]
  
  \begin{framed}
  {\raggedright
      \small
  
  \begin{hangparas}{1em}{1}
        \hrule
        \vspace{0.15cm}
        \textsc{\textbf{Protocol $\protocolNumber$: In-Person Trust Establishment}}
        \vspace{0.1cm}
        \hrule
        \vspace{0.1cm}
  \medskip
      
      \textbf{Encode.}
          Alice encodes her key exchange public key $\kx_A^P$ and her allocation index $i_A$ into a human-readable story $s_A$. Bob similarly encodes his story $s_B$.

  \medskip

      \textbf{Share.}
          Alice and Bob meet. They share their stories, and type each other's story into their own client. 

          \medskip

      \textbf{Establish contact.}
          Alice decodes Bob's story to obtain $\kx^P_B$ and $i_B$. She computes the shared secret $k = \DH(\kx^P_B, \kx^S_A)$, and adds $i_B$ to her set of listening indices.


  \end{hangparas}
  }
  \end{framed}
  \caption{Establishing trust when the participants can talk to each other.}
  \label{fig:trust-establishment-inperson}
\end{figure}

This trust establishment protocol needs no third party, which means that it completes instantly, cheaply, and securely. It does, however, require interaction from both parties at the same time. While it may seem prohibitive to set up a meeting and type manually, we believe that it is worth it for security-critical use cases.

%%%%%%%%%%%%%%%%%%%%%%%%%%%%%%%%%%%%%%%%%%%%%%%%%%%%%%%%%%%%%%%
%%%%% Async Section %%%%%
%%%%%%%%%%%%%%%%%%%%%%%%%%%%%%%%%%%%%%%%%%%%%%%%%%%%%%%%%%%%%%%

\begin{figure}[th!]
  \begin{framed}
  {\raggedright
      \small
  
  \begin{hangparas}{1em}{1}
        \hrule
        \vspace{0.15cm}
        \textsc{\textbf{Protocol $\protocolNumber$: Asynchronous Trust Establishment}}
        \vspace{0.1cm}
        \hrule
        \vspace{0.1cm}
  \medskip
      
      \textbf{Send invitation.}
      Alice initiates contact with Bob.
      \begin{enumerate}
          \item \textbf{Visibility.} Bob computes a ``public id" $\id_B = (\ki_B^P, \kx_B^P, i_B)$, where $\ki_B^P$ is the invitation public key, $\kx_B^P$ is the key exchange public key, and $i_B$ is the outbox index. Bob shares $id_B$ publicly.
          \item \textbf{Invitation.} Alice drafts a message $m_{AB}$ for Bob and obtains the triple $(\ki_B^P, \kx_B^P, i_B)$ from Bob's profile. She periodically sends the pair $(i_A, c_{AB} = \Enc_{\kx_B^P}(\id_A \vert m_{AB}))$ to the server, which stores it in a database$~$$\db_\text{async}$. When Alice has no invitations, she sends$~$$(i_A, \Enc_{\kx^P}(\id_A | m))$, where $\kx^P$ and $m$ are random.
          \item \textbf{Invitation message.} Alice computes a shared secret $\sk =  \DH(\kx_B^P, \kx^S_A)$ with Bob. She then sends a message $\inv_{AB}$ via our PIR transport layer, inviting Bob to join a conversation with her. She waits for Bob's ACK.
      \end{enumerate}

  \medskip

      \textbf{Retrieve invitation.}
        Bob receives Alice's invitation.
        \begin{enumerate}
            \item \textbf{Database download.} Bob periodically downloads the entire $\db_\text{async}$.
            \item \textbf{Find invitations.} Bob computes $\Dec_{\kx^S_B}(c)$ over all key-value pairs $(i, c)$ in $\db_{\text{async}}$. If the decryption fails, Bob ignores this pair. If the decryption succeeds for the pair $(i, c) = (i_A, c_{AB})$, Bob decodes $i = i_A, \id_A, m_{AB}$, and records an invitation for him.
            \item \textbf{Verify.} Bob needs to verify Alice's identity using $\id_A$ and $m_{AB}$, say by checking Alice's profile. Bob now chooses to accept or reject the invitation.
        \end{enumerate}

    \medskip

      \textbf{Accept invitation.}
          If Bob accepts, he responds to Alice.
          \begin{enumerate}
              \item \textbf{Decode.} Bob decodes $\id_A = (\ki_A^P, \kx_A^P, i_A)$. He computes the shared secret $k =  \DH(\kx_B^P, \kx^S_A)$ with Alice. He adds $i_A$ to the list of indices he listens to.
              \item \textbf{ACK.} Bob recieves the invitation message from Alice, $\inv_{AB}$. Bob ACKs the message.
              \item \textbf{ACK received.} Alice sees the ACK, finalizing the trust establishment.
          \end{enumerate}
    \medskip
          
      \textbf{Reject invitation.}
        If Bob rejects, no further action is needed.
  \end{hangparas}
  }
  \end{framed}
  \caption{Establishing trust remotely. Contact can be initiated only by a single participant and requires acknowledgement from both participants.}
  \label{fig:trust-establishment-async}
\end{figure}


\subsection{Asynchronous Invitations}

Alice and Bob can also establish trust asynchronously. Here, we assume that Alice knows Bob's \textit{public identifier}, $\id_B$. For example, Bob may have posted his public identifier on a public platform like Twitter or his website. Alice can use this $\id_B$ to privately send an invitation, which Bob can later accept or reject. 

To send the invitation, Alice encrypts her own $\id_A$ using Bob's public key and sends the result to the server. Bob then downloads the entire database of invitations, and finds the invitations meant for him by trying to decrypt each one of them. The full protocol is described in \cref{fig:trust-establishment-async}.

This method achieves metadata privacy by hiding the timing of the invitation and the recipient of Alice's request. The client sends invitations on a fixed schedule, and for the public-key cryptography we use an ElGamal-based algorithm, which satisfies key-privacy \cite{bellare2001key}.
    
This method offers trust establishment with convenience on par with existing messaging platforms, at the expense of increased bandwidth costs. A $\db_\text{async}$ entry requires $\approx200$ B of storage, which for 10 million users is $2$ GB for the entire database. An individual user might wish to download the database only once or twice daily, which saves bandwidth but delays the detection of invitations. On the other hand, a company user could afford to download the database every few seconds, to ensure that incoming asynchronous invitations get detected almost instantly.

With an optimization, we can make detection even more accessible. Instead of downloading the whole database, the server streams a log of changes to $\db_{\text{sync}}$ to each user. If each user writes to $\db_{\text{sync}}$ every 20 minutes, then every second only $1 / 1200$ of the 10 million users will change their entries, which corresponds to a change log of less than $2 \text{MB}$. Thus, any user with download bandwidth at least $2 \text{MB / s}$ can detect incoming asynchronous invitations instantly.

The asynchronous method introduces more attack vectors. For example, an attacker can socially engineer their way to make Alice believe that a fake $\id_B$ belongs to Bob. Note that the identifier does not include Bob's name, which we believe reduces the potential of this attack, because it makes Alice more likely to verify the identifier before accepting it. An attacker could also monitor Alice's traffic to wherever Bob posts his public identifier. In conclusion, we provide the asynchronous method as a convenient choice, but we recommend that security-critical users use in-person invitations whenever possible. Still, to our knowledge, our asynchronous method is more private than any other asynchronous trust establishment that exists in practice.