\section{Trust Establishment}
\label{sec:trustestablishment}

Most existing metadata private messaging systems, such as Pung or Addra, assume that users know beforehand who they wish to talk to, and had a prior key exchange through another channel. In our messaging application, we need a metadata-secure mechanism for the key exchange itself. In other words, if Alice knows the public key $pk_B$ of Bob, then Alice should be able to send an ``invitation" to Bob. Bob must be able to retrieve this invitation from the server, and complete a key exchange with Alice. We call this process ``trust establishment".

This problem is known as Oblivious Message Detection(OMD) in \cite{liutromer2021}. The scheme proposed in \cite{liutromer2021} aims to minimize user download size, but it costs each user $\$ 1$ per million messages scanned, which is prohibitive for our messaging application. We provide two alternate methods of trust establishment with better computational cost and security.

In the following two methods, we assume that Alice and Bob's daemons have generated a Curve25519 key exchange keypair $kx = (kx^P, kx^S)$. Their goal is to inform each other of their $kx^P$ to derive a shared secret $sk$.

\subsection{Face-to-face Invitations}
Our first method assumes that Alice and Bob are able to set up a face-to-face meeting with each other, either in person or over zoom. The trust establishment process is a simple key exchange implemented as follows.

1. Alice encodes the plaintext of her key exchange public key $kx_A^P$ and her allocation index $i_A$ into a human-readable story $s_A$. Bob similarly encodes his story $s_B$.

2. Alice and Bob meet. They share their stories, and type the other's story into their own client. 

3. Alice decodes Bob's story to obtain $kx^P_B$ and $i_B$. She computes the shared secret $sk = DH(kx^P_B, kx^S_A)$, and adds $i_B$ to her set of listening indices.\todo{Better name?} Bob does the same. They can now communicate to each other using our PIR transport layer.

Using this method, all interactions between the users do not pass through any third party. Thus, trust establishment can be completed instantly, cheaply, and securely. Our system do require Alice and Bob to be able to set up a meeting and type each other's story manually. To justify this approach, we believe a privacy-conscious Alice would be willing to set up a face-to-face meeting with Bob before sending him sensitive information.

\subsection{Asynchronous Invitation}

\todo{mention key privacy}

Our second method targets the case when Bob does not know about Alice's invitation beforehand. For example, Alice can be a sensitive client who wishes to privately reach out to Bob's company. 

This method consists of three steps. First, Alice sends an encrypted invitation to the server without indicating its recipient. Second, Bob retrieves this invitation via a full database download. Third, Bob informs Alice of his acceptance via an ACK message.

\textbf{To send an invitation}

1. Upon registration, each user's daemon generates an additional invitation keypair $ki = (ki^P, ki^S)$. Bob's daemon computes a ``public id" $id_B$, which contains his invitation public key $ki_B^P$, key exchange public key $kx_B^P$, and allocation $i_B$ encoded in plaintext. It then displays the id on Bob's GUI. Bob posts $id_B$ on his public profile, such as on twitter or on his company's website.

3. When Alice wishes to send an invitation to Bob, she obtains $id_B$ from Bob's public profile, and decodes it to obtain $(i_B, kx_B^P, ki_B^P)$. She drafts an initial message $m_{AB}$ to accompany her invitation. 

4. Alice's daemon periodically sends the key-value pair $(i_A, c_{AB} = \Enc(kx_B^P, id_A \vert m_{AB}))$ to the server\footnote{It is important to redo this encryption each round, otherwise adversaries will observe the same message repeatedly.}, which stores it in a separate \textbf{AsyncInvitationDatabase}. When Alice has no invitations, her daemon sends $(i_A, \Enc(kx^P, id_A))$ for a random public key $kx^P$.

5. As Alice's daemon sends the invitation, it also compute the shared secret with Bob $sk = DH(kx_B^P, kx^S_A)$. It sends Bob a ``control message" $ctm_{AB}$ via our PIR transport layer, and listens for Bob's ACK.

\textbf{To retrieve an invitation}

1. Bob's daemon periodically downloads the entire \textbf{AsyncInvitationDatabase}. It computes $\Dec(kx^S_B, c)$ over all key-value pairs $(i, c)$. If the decryption fails, Bob's daemon ignores this pair. Now suppose $(i, c) = (i_A, c_{AB})$, and the decryption succeeds. Bob's daemon decodes $i = i_A, id_A, m_{AB}$ from the decrypted data, and displays on Bob's GUI that he received an incoming invitation from $id_A$ with message $m_{AB}$.

2. Bob verifies Alice's identity using $id_A$ and $m_{AB}$, for example by checking Alice's public profile. Bob then chooses to either accept or reject the invitation. If Bob rejects the invitation, no further action is performed. 

\textbf{To accept an invitation}

1. If Bob accepts Alice's invitation, Bob's daemon decodes $id_A$ to obtain $kx_A^P$, and computes the shared secret $sk = DH(kx_A^P, kx_B^S)$. It adds $i_A$ to the set of listening indices.

2. Since Alice is sending the control message $ctm_{AB}$ to Bob using the same shared secret $sk$, Bob's daemon will read $ctm_{AB}$ from the PIR database. It sends an ACK to Alice's control message.

3. When Alice's daemon reads Bob's ACK to $ctm_{AB}$ from the PIR ACK database, it displays on Alice's GUI that Bob has accepted Alice's invitation. Alice and Bob can now communicate to each other using our PIR transport layer.

This method achieves metadata security. We hide the timing of the invitation by making the daemon send to and download \textbf{AsyncInvitationDatabase} on a fixed schedule. We hide the recipient of Alice's request since our encryption scheme is key private(\todo{Cite arvid's section}). 

This method offers convenient trust establishment on par with most existing messaging platforms. Its disadvantage is that downloading the entire database is expensive. We estimate that a key-value pair in the \textbf{AsyncInvitationDatabase} take approximately $200\text{B}$. If we have 1 million users, then the whole database will have a size of approximately $200\text{MB}$. Thus, an individual user might wish to only download the database once or twice each day, which saves bandwidth but delays the detection of invitations. On the other hand, a company user could afford downloading the database every second, which takes $200\text{MB}/s$ download bandwidth but can ensure incoming asynchronous invitations get detected almost instantly.

Finally, we note that this method introduces more attack vectors. For example, an attacker might use social engineering to make Alice believe that a fake $id_B$ belongs to Bob, or simply monitor $A$'s traffic to $B$'s public profile. Therefore, we recommend privacy conscious users to use face-to-face invitations whenever possible.